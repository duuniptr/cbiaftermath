%%%%%%%%%%%%%%%%%%%%%%%%%%%%%%%%%%%%%%%%%%%%%%%%%%%%%%%%%%%%%%%%%%%%
%%%%%%%%%%%%%%%%%%%%%%%%%%%%%%%%%%%%%%%%%%%%%%%%%%%%%%%%%%%%%%%%%%%%
%%                                                                %%
%% Esimerkki opinnäytteen tekemisestä LaTeX:lla 20120320          %%
%% Alkuperäinen versio Luis Costa,  muutokset Perttu Puska        %%
%%                                                                %%
%% Tähän esimerkkiin kuuluu tiedostot                             %%
%%               opinnaytepohja.tex (versio 1.6)                  %%
%%               aaltothesis.sty (versio 1.6)                     %%
%%               kuva1.eps                                        %%
%%               kuva2.eps                                        %%
%%                                                                %%
%%                                                                %%
%% Kääntäminen                                                    %%
%% latex:                                                         %%
%%             $ latex opinnaytepohja                             %%
%%             $ latex opinnaytepohja                             %%
%%                                                                %%
%%   Tuloksena on tiedosto opinnayte.dvi, joka                    %%
%%   muutetaan ps-muotoon seuraavasti                             %%
%%                                                                %%
%%             $ dvips opinnaytepohja -o                          %%
%%                                                                %%
%% Selittävät kommentit on tässä esimerkissä varustettu           %%
%% %%-merkeillä ja muutokset, joita käyttäjä voi tehdä,           %%
%% on varustettu %-merkeillä                                      %%
%%                                                                %%
%%%%%%%%%%%%%%%%%%%%%%%%%%%%%%%%%%%%%%%%%%%%%%%%%%%%%%%%%%%%%%%%%%%%
%%%%%%%%%%%%%%%%%%%%%%%%%%%%%%%%%%%%%%%%%%%%%%%%%%%%%%%%%%%%%%%%%%%%

%% Käytä toinen näistä, jos kirjoitat suomeksi:
%% ensimmäinen, jos käytät pdflatexia (kuvat on oltava pdf-tiedostoina)
%% toinen, jos haluat tuottaa ps-tiedostoa (käytä eps-formaattia kuville).
%%
%% Use one of these you write in Finnish:
%% the 1st when using pdflatex (use pdf figures) or
%% the 2nd when producing a ps file (use eps figures).
%\documentclass[finnish,12pt,a4paper,pdftex]{article}
%\documentclass[finnish,12pt,a4paper,dvips]{article}

%% Käytä näitä, jos kirjoitat englanniksi
%%
%% Uncomment one of these if you write in English
%\documentclass[english,12pt,a4paper,pdftex]{article}
\documentclass[english,12pt,a4paper,dvips]{article}

%% Nämä komennot asettavat oikean tekstiencodauksen.
%%\usepackage[utf8]{inputenc}

\usepackage[utf8]{inputenc}
\usepackage[OT1]{fontenc}
\usepackage[finnish]{babel}

%% Tämä paketti on pakollinen
%% Valitse korkeakoulusi näistä: arts, biz, chem, elec, eng, sci.
%%
%% This package is required
%% Choose your school from arts, biz, chem, elec, eng, sci.
\usepackage[elec]{aaltothesis}

%% Jos käytät latex-komentoa käännettäessä (oletusarvo) 
%% kuvat kannattaa tehdä eps-muotoon. Älä käytä ps-muotoisia kuvia!
%% Käytä seuraavaa latex-komennon ja eps-kuvien kanssa 
%%
%% Jos tääs käytät pdflatex-komentoa, joka kääntää tekstin suoraan
%% pdf-tiedostoksi, kuvasi on oltava jpg-formaatissa tai pdf-formaatissa.
%%
%% Use this if you run pdflatex and use jpg/pdf-format pictures.
%%
\usepackage{graphicx}

%% Saat pdf-tiedoston viittaukset ja linkit kuntoon seuraavalla paketilla.
%% Paketti toimii erityisen hyvin pdflatexin kanssa. 
%%
%% Use this if you want to get links and nice output with pdflatex
\usepackage[pdfpagemode=None,colorlinks=true,urlcolor=red,%
linkcolor=blue,citecolor=black,pdfstartview=FitH]{hyperref}

%% Jos et jostain syystä tykkää käyttää
%% edellistä hyperref pakettia, voit käyttää myös seuraavaa pakettia
%% (tarvitaan lähinnä url-komennon määrittämiseen ja formatoimiseen)
%%
%% Use this if you do not like hyperref package - this
%% defines url environment and formats it correctly
\usepackage{url}

%% Matematiikan fontteja, symboleja ja muotoiluja lisää, näitä tarvitaan usein 
%%
%% Use this if you write hard core mathematics, these are usually needed
\usepackage{amsfonts,amssymb,amsbsy}  

%% Vaakasuunnan mitat, ÄLÄ KOSKE!
\setlength{\hoffset}{-1in}
\setlength{\oddsidemargin}{35mm}
\setlength{\evensidemargin}{25mm}
\setlength{\textwidth}{15cm}
%% Pystysuunnan mitat, ÄLÄ KOSKE!
\setlength{\voffset}{-1in}
\setlength{\headsep}{7mm}
\setlength{\headheight}{1em}
\setlength{\topmargin}{25mm-\headheight-\headsep}
\setlength{\textheight}{23cm}


%%\titleformat{\section}%
%%\renewcommand{\thesubsection}{\arabic{subsection}}%... from subsections


\setcounter{secnumdepth}{5}
\setcounter{tocdepth}{5}
%% Kaikki mikä paperille tulostuu, on tämän jälkeen
%%
%% Output starts here
\begin{document}

%% Korjaa vastaamaan korkeakouluasi, jos automaattisesti asetettu nimi on 
%% virheellinen 
%%
%% Change the school field to describe your school if the autimatically 
%% set name is wrong
% \university{aalto University}{aalto-Yliopisto}
% \school{School of Electrical Engineering}{SähköTekniikan korkeakoulu}

%% Vain kandityölle: Korjaa seuraavat vastaamaan koulutusohjelmaasi
%%
%% Only for B.Sc. thesis: Choose your degree programme. 
\degreeprogram{Electronics and electrical engineering}{Elektroniikka ja sähkötekniikka}
%%

%% Vain DI/M.Sc.- ja lisensiaatintyölle: valitse laitos, 
%% professuuri ja sen professuurikoodi. 
%%
%% Only for M.Sc. and Licentiate thesis: Choose your department,
%% professorship and professorship code. 
%\department{Department of Radio Science and Technology}%
%{Radiotieteen ja -tekniikan laitos}
%\professorship{Circuit theory}{Piiriteoria}
%\code{S-55}
%%

%% Valitse yksi näistä kolmesta
%%
%% Choose one of these:
\univdegree{BSc}
%\univdegree{MSc}
%\univdegree{Lic}

%% Oma nimi
%%
%% Should be self explanatory...
\author{Peter Tapio}

%% Opinnäytteen otsikko tulee vain tähän. Älä tavuta otsikkoa ja
%% vältä liian pitkää otsikkotekstiä. Jos latex ryhmittelee otsikon
%% huonosti, voit joutua pakottamaan rivinvaihdon \\ kontrollimerkillä.
%% Muista että otsikkoja ei tavuteta! 
%% Jos otsikossa on ja-sana, se ei jää rivin viimeiseksi sanaksi 
%% vaan aloittaa uuden rivin.
%% 
%% Your thesis title. If the title is very long and the latex 
%% does unsatisfactory job of breaking the lines, you will have to
%% break the lines yourself with \\ control character. 
%% Do not hyphenate titles.
\thesistitle{Limitations of video conferencing technology}

\place{Espoo}
%% Kandidaatintyön päivämäärä on sen esityspäivämäärä! 
%% 
%% For B.Sc. thesis use the date when you present your thesis. 
\date{1.5.2014}

%% Kandidaattiseminaarin vastuuopettaja tai diplomityön valvoja.
%% Huomaa tittelissä "\" -merkki pisteen jälkeen, 
%% ennen välilyöntiä ja seuraavaa merkkijonoa. 
%% Näin tehdään, koska kyseessä ei ole lauseen loppu, jonka jälkeen tulee 
%% hieman pidempi väli vaan halutaan tavallinen väli.
%%
%% B.Sc. or M.Sc. thesis supervisor 
%% Note the "\" after the comma. This forces the following space to be 
%% a normal interword space, not the space that starts a new sentence. 
\supervisor{Prof.\ Markus Turunen}{Prof.\ Markus Turunen}

%% Kandidaatintyön ohjaaja(t) tai diplomityön ohjaaja(t)
%% 
%% B.Sc. or M.Sc. thesis advisors(s). 
%%
%% Note that there has been a change in the official EN translation
%% of the Finnish title ``ohjaaja'' which in the previous version (1.5) 
%% of this document was called ``instructor''. The recommended
%% translation is now ``advisor''.  
%% However, the LaTeX internal variable remains \instructor
%% as there is little point to change the variable name. 
%%
%\instructor{Prof. Pirjo Professori}{Prof. Pirjo Professori}
%\instructor{D.Sc.\ (Tech.) Olli Ohjaaja}{TkT Olli Ohjaaja}
%\instructor{M.Sc.\ (Tech.) Polli Pohjaaja}{DI Polli Pohjaaja}
\instructor{Maria Clavert}{Maria Clavert}%
\instructor{Joona Kurikka}{Joona Kurikka}%
\instructor{Elina Kahkonen}{Elina Kahkonen}%
\instructor{Tuomas Paloposki}{Tuomas Paloposki}%

%% Aaltologo: syntaksi:
%% \uselogo{aaltoRed|aaltoBlue|aaltoYellow|aaltoGray|aaltoGrayScale}{?|!|''}
%% Logon kieli on sama kuin dokumentin kieli
%%
%% Aalto logo: syntax:
% \uselogo{aaltoRed|aaltoBlue|aaltoYellow|aaltoGray|aaltoGrayScale}{?|!|''}
%% Logo language is set to be the same as the document language.
\uselogo{aaltoBlue}{?}

%% Tehdään kansilehti
%%
%% Create the coverpage
\makecoverpage

%% English abstract, uncomment if you need one. 
%% 
%% Abstract keywords
\keywords{webrtc, teleconferencing,\\ product, development}
%% Abstract text
\begin{abstractpage}[english]

 This bachelors thesis is based on a multidisciplinary project based pilot course, done partially at CERN to develop new technologies for education. The findings and point of view of the multidiciplinary team and the relevance of those findings to the final technical result are explained and evaluated. The final prototype, related to teleconferencing, is explained and the technical properties are analysed.
 
\end{abstractpage}
%% Note that 
%% if you are writting your master's thesis in English place the English
%% abstract first followed by the possible Finnish abstract
%% Suomenkielinen tiivistelmä
%% 
%% Finnish abstract
%%
%% Tiivistelmän avainsanat
\keywords{webrtc, videoneuvottelu,\\ poikkitieteellinen, tuotekehitys}
%% Tiivistelmän tekstiosa
\begin{abstractpage}[finnish]
  Tässä työssä käsitellään kaikenlaista jännää....
\end{abstractpage}

%% Pakotetaan uusi sivu varmuuden vuoksi, jotta 
%% mahdollinen suomenkielinen ja englanninkielinen tiivistelmä
%% eivät tule vahingossakaan samalle sivulle
%%
%% Force new page so that English abstract starts from a new page
\newpage
%


%% Esipuhe 
%%
%% Preface
%\mysection{Esipuhe}
\mysection{Preface}
I'd like to thank...

%Haluan kiittää Professori Pirjo 
%Professoria ja ohjaajaani Olli Ohjaajaa hyvästä ja 
%huonosta ohjauksesta.\\

\vspace{5cm}
Otaniemi, 9.3.2014

\vspace{5mm}
{\hfill Peter W.\ Tapio \hspace{1cm}}

%% Pakotetaan varmuuden vuoksi esipuheen jälkeinen osa
%% alkamaan uudelta sivulta
%%
%% Force new page after preface
\newpage


%% Sisällysluettelo
%% addcontentsline tekee pdf-tiedostoon viitteen sisällysluetteloa varten
%% 
%% Table of contents. 
%\addcontentsline{toc}{section}{Sisällysluettelo}
\addcontentsline{toc}{section}{Contents}
%% Tehdään sisällysluettelo
%%
%% Create it. 
\tableofcontents


%% Symbolit ja lyhenteet
%%
%% Symbols and abbreviations
%\mysection{Symbolit ja lyhenteet}
\section{Abbreviations}

\begin{tabular}{ll}
CERN        & European Organization for Nuclear Research \\
CBI         & Challenge Based Innovaition course at CERN,\\
            & in collaboration with various universities \\
TCP         & Transfer Control Protocol \\ %%
UDP         & User Datagram Protocol \\
HTML5       & Hyper Text Markup Language version 5 \\
WebRTC      & Web Real Time Communication \\
IBL         & ATLAS Insertable B-Layer detector 
\end{tabular}


%% Sivulaskurin viilausta opinnäytteen vaatimusten mukaan:
%% Aloitetaan sivunumerointi arabialaisilla numeroilla (ja jätetään
%% leipätekstin ensimmäinen sivu tyhjäksi, 
%% ks. alla \thispagestyle{empty}).
%% Pakotetaan lisäksi ensimmäinen varsinainen tekstisivu alkamaan 
%% uudelta sivulta clearpage-komennolla. 
%% clearpage on melkein samanlainen kuin newpage, mutta 
%% flushaa myös LaTeX:n floatit 
%% 
%% Corrects the page numbering, there is no need to change these
\cleardoublepage
\storeinipagenumber
\pagenumbering{arabic}
\setcounter{page}{1}


%% Leipäteksti alkaa
%%
%% Text body begins. Note that since the text body
%% is mostly in Finnish the majority of comments are
%% also in Finnish after this point. There is no point in explaining
%% Finnish-language specific thesis conventions in English.
%\section{Johdanto}
\section{Introduction}

%% Ensimmäinen sivu tyhjäksi
%% 
%% Leave first page empty
\thispagestyle{empty}

This work is based on a project done at the European Organization for Nuclear Research (CERN) for the TALENT project, that is a part of the Marie Curie Initial Training Network. The purpose of this thesis is to elaborate the background of the prototype produced for the  TALENT project, to explain how the needs of the user affected the selection of the technological solution and elaborate on the input given by the multidisciplinary team members assigned to the project.
The technologies and standards relevant to the project will be explored and evaluated against the findings of the student project. In the end, the strengths and weaknesses of the selected technology will be discussed and other possible solutions will be presented for comparison.


%% Opinnäytteessä jokainen osa alkaa uudelta sivulta, joten \clearpage
%%
%% In a thesis, every section starts a new page, hence \clearpage
\clearpage

%\section{Aikaisempi tutkimus}
\section{The CBI project for TALENT at CERN}

This chapter briefly explains the nature of the project course in question and its relation to CERN and the TALENT project. The connection to research and development activities at CERN is an important influence on the decisions made during the project and provides a point of view also in this work. 


\subsection*{The Challenge Based Innovation course}

The Challenge Based Innovation (CBI) course, organized 2013-2014, was a pilot product development course..

(need to find course leaflet and extra info)
(List participating universities)

\subsection*{The TALENT project}

The TALENT project at CERN, has young researchers in the field of instrumentation for radiation detection, developing new state-of-the-art technologies for the ATLAS detector. The project involves people from various countries and universities, working with multidisciplinary industry in the various fields of advanced radiation sensors, fast and efficient data acquisition electronics, new cooling technologies and ultra light support structures.


\subsection*{The project brief for the CBI team}

The two main requirements for the project were a connection to TALENT and a relation to education. The brief was left fairly open, so that the multidisciplinary team could customize it to suit their specific skillset. On a more detailed level, three different tracks were presented for the team to choose.

\begin{itemize}
\item[--]Using TALENT technologies for education.
\item[--]Educating a focus group about the TALENT technologies and the science behind the various technologies.
\item[--]To develop a technology that supports the research activities of TALENT.
\end{itemize}

The final problem definition and type of solution was left for the student team to decide and discuss with members of the TALENT project. A design thinking approach would then be used to further determine the final nature of the project.

\clearpage

\section{Scope of the project and reasoning for the final concept}
- Briefly about finding direction for the project

\subsection*{Needfinding}
- Methods used
- Needs found

\subsection*{Final concept}
- Overview

\clearpage

\section{Final prototype for the concept}
- The prototype is a subset of functions of the final concept

\subsection*{Requirements}
- Practical matters

\subsubsection*{Technical limitations}
- Network
- Platform compatibility

\subsubsection*{Other limitations}
- Time to prototype
- Resources

\subsection*{Available options for the technological solution}
- Overview
- Reasoning

\subsubsection*{Selected frontend software}
(Chrome, HTML5, javascript)

\subsubsection*{WebRTC}


\subsubsection*{Selected backend software}
(nginx, nodeJS..)

\clearpage

\section{Description of prototype}
5.1 User centered design
5.2 Platform independence


\clearpage

\section{Conclusion}
\clearpage

%% Lähdeluettelo
%\addcontentsline{toc}{section}{Viitteet}
\addcontentsline{toc}{section}{References}

\begin{thebibliography}{99}

\bibitem{Adeyeye} Adeyeye,\ M., Makitla,\ I., Fogwill,\ T. (2013) \textit{Determining the signalling overhead of two common WebRTC methods: JSON via XMLHttpRequest and SIP over WebSocket} IEEE 2013

\bibitem{Jennings} Jennings,\ C., Hardie,\ T., Westerlund, M. (2013) \textit{Real-time communications for the web} Communications Magazine, IEEE Vol. 51, Iss. 4, pp. 20-26 

\bibitem{Garaizar} Garaizar,\ P., Vadillo,\ M.\ A., Lopez-de-Ipina,\ D. (2012) \textit{Benefits and pitfalls of using HTML5 APIs for online experiments and simulations} Remote Engineering and Virtual Instrumentation (REV), 2012 9th International Conference on pp. 1-7. ISBN: 978-1-4673-2540-0

\bibitem{Hunger} Hunger,\ A., Hirlehei,\ A. (2013) \textit{Supporting globally distributed work - Cultural adaptivity meets groupware tailorability} Lecture Notes in Computer Science (including subseries Lecture Notes in Artificial Intelligence and Lecture Notes in Bioinformatics) 8024 LNCS: (PART 2) p. 219-227 2013. ISBN: 9783642391361

\bibitem{Mootee} Mootee,\ I. (2013) \textit{Design thinking for strategic innovation : what they can't teach you at business or design school} Hoboken: Wiley , 2013. (e-book) ISBN: 9781118748855

\bibitem{} \textit{}


\end{thebibliography}

%% Liitteet 
\appendix 

\clearpage
\addcontentsline{toc}{section}{Liite A}
\section{CBI course description\label{LiiteA}}
%% Liitteiden kaavat, taulukot ja kuvat numeroidaan omana kokonaisuutenaan
%%
%% Equations, tables and figures have their own numbering in Appendices
\renewcommand{\theequation}{A\arabic{equation}}
\setcounter{equation}{0}  
\renewcommand{\thefigure}{A\arabic{figure}}
\setcounter{figure}{0}
\renewcommand{\thetable}{A\arabic{table}}
\setcounter{table}{0}

Liitteet eivät ole opinnäytteen kannalta välttämättömiä ja 
opinnäytteen tekijän on 
kirjoittamaan ryhtyessään hyvä ajatella pärjäävänsä ilman liitteitä.
Kokemattomat kirjoittajat, jotka ovat huolissaan
tekstiosan pituudesta, paisuttavat turhan 
helposti liitteitä pitääkseen tekstiosan pituuden annetuissa rajoissa.
Tällä tavalla ei synny hyvää opinnäytettä.   

Liite on itsenäinen kokonaisuus, vaikka se täydentääkin tekstiosaa.
Liite ei siten ole pelkkä listaus, kuva tai taulukko, vaan 
liitteessä selitetään aina sisällön laatu ja tarkoitus. 

Liitteeseen voi laittaa esimerkiksi listauksia. Alla on 
listausesimerkki tämän liitteen luomisesta. 

%% Verbatim-ympäristö ei muotoile tai tavuta tekstiä. Fontti on monospace.
%% Verbatim-ympäristön sisällä annettuja komentoja ei LaTeX käsittele. 
%% Vasta \end{verbatim}-komennon jälkeen jatketaan käsittelyä.
\begin{verbatim}
	\clearpage
	\appendix
	\addcontentsline{toc}{section}{Liite A}
	\section*{Liite A}
	...
	\thispagestyle{empty}
	...
	tekstiä
	...
	\clearpage
\end{verbatim}



\clearpage
\addcontentsline{toc}{section}{Liite B}
\section{Project brief from TALENT\label{LiiteB}}

%% Liitteiden kaavat, taulukot ja kuvat numeroidaan omana kokonaisuutenaan
%%
%% Equations, tables and figures have their own numbering in Appendices
\renewcommand{\theequation}{B\arabic{equation}}
\setcounter{equation}{0}  
\renewcommand{\thefigure}{B\arabic{figure}}
\setcounter{figure}{0}
\renewcommand{\thetable}{B\arabic{table}}
\setcounter{table}{0}

Liitteissä voi myös olla kuvia, jotka
eivät sovi leipätekstin joukkoon:
%% Ympäristön figure parametrit htb pakottavat
%% kuvan tähän, eikä LaTeX yritä siirrellä niitä
%% hyväksi katsomaansa paikkaan. 
%% Ympäristöä center voi käyttää \centering-
%% komennon sijaan
%%
%% Example of a figure, note the use of htb parameters which force
%% the figure to be inserted here
\begin{figure}[htb]
\begin{center}
\includegraphics[height=8cm]{kuva2}
\end{center}
\caption{Kuvateksti, jossa on liitteen numerointi \label{liitekuva}}
\end{figure}
%%
Liitteiden taulukoiden numerointi on kuvien ja kaavojen kaltainen:
\begin{table}[htb]
\caption{Taulukon kuvateksti. \label{liitetaulukko}}
\begin{center}
\fbox{
\begin{tabular}{lp{0.5\linewidth}}
9.00--9.55  & Käytettävyystestauksen tiedotustilaisuus (osanottajat
ovat saaneet sähköpostitse valmistautumistehtävät, joten tiedotustilaisuus
voidaan pitää lyhyenä).\\
9.55--10.00 & Testausalueelle siirtyminen
\end{tabular}}
\end{center}
\end{table}
Kaavojen numerointi muodostaa liitteissä oman kokonaisuutensa:
\begin{eqnarray}
T_{ik} &=& -p g_{ik} + w u_i u_k + \tau_{ik},  \label{liitekaava3} \\
n_i    &=& n u_i + v_i.                        \label{liitekaava4}
\end{eqnarray}

\end{document}


