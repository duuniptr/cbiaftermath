%%%%%%%%%%%%%%%%%%%%%%%%%%%%%%%%%%%%%%%%%%%%%%%%%%%%%%%%%%%%%%%%%%%%
%%%%%%%%%%%%%%%%%%%%%%%%%%%%%%%%%%%%%%%%%%%%%%%%%%%%%%%%%%%%%%%%%%%%
%%                                                                %%
%% Esimerkki opinnäytteen tekemisestä LaTeX:lla 20120320          %%
%% Alkuperäinen versio Luis Costa,  muutokset Perttu Puska        %%
%%                                                                %%
%% Tähän esimerkkiin kuuluu tiedostot                             %%
%%               opinnaytepohja.tex (versio 1.6)                  %%
%%               aaltothesis.sty (versio 1.6)                     %%
%%               kuva1.eps                                        %%
%%               kuva2.eps                                        %%
%%                                                                %%
%%                                                                %%
%% Kääntäminen                                                    %%
%% latex:                                                         %%
%%             $ latex opinnaytepohja                             %%
%%             $ latex opinnaytepohja                             %%
%%                                                                %%
%%   Tuloksena on tiedosto opinnayte.dvi, joka                    %%
%%   muutetaan ps-muotoon seuraavasti                             %%
%%                                                                %%
%%             $ dvips opinnaytepohja -o                          %%
%%                                                                %%
%% Selittävät kommentit on tässä esimerkissä varustettu           %%
%% %%-merkeillä ja muutokset, joita käyttäjä voi tehdä,           %%
%% on varustettu %-merkeillä                                      %%
%%                                                                %%
%%%%%%%%%%%%%%%%%%%%%%%%%%%%%%%%%%%%%%%%%%%%%%%%%%%%%%%%%%%%%%%%%%%%
%%%%%%%%%%%%%%%%%%%%%%%%%%%%%%%%%%%%%%%%%%%%%%%%%%%%%%%%%%%%%%%%%%%%

%% Käytä toinen näistä, jos kirjoitat suomeksi:
%% ensimmäinen, jos käytät pdflatexia (kuvat on oltava pdf-tiedostoina)
%% toinen, jos haluat tuottaa ps-tiedostoa (käytä eps-formaattia kuville).
%%
%% Use one of these you write in Finnish:
%% the 1st when using pdflatex (use pdf figures) or
%% the 2nd when producing a ps file (use eps figures).
%\documentclass[finnish,12pt,a4paper,pdftex]{article}
%\documentclass[finnish,12pt,a4paper,dvips]{article}

%% Käytä näitä, jos kirjoitat englanniksi
%%
%% Uncomment one of these if you write in English
%\documentclass[english,12pt,a4paper,pdftex]{article}
\documentclass[english,12pt,a4paper,dvips]{article}

%% Nämä komennot asettavat oikean tekstiencodauksen.
%%\usepackage[utf8]{inputenc}

\usepackage[utf8]{inputenc}
\usepackage[OT1]{fontenc}
\usepackage[finnish]{babel}

%% Tämä paketti on pakollinen
%% Valitse korkeakoulusi näistä: arts, biz, chem, elec, eng, sci.
%%
%% This package is required
%% Choose your school from arts, biz, chem, elec, eng, sci.
\usepackage[elec]{aaltothesis}

%% Jos käytät latex-komentoa käännettäessä (oletusarvo) 
%% kuvat kannattaa tehdä eps-muotoon. Älä käytä ps-muotoisia kuvia!
%% Käytä seuraavaa latex-komennon ja eps-kuvien kanssa 
%%
%% Jos tääs käytät pdflatex-komentoa, joka kääntää tekstin suoraan
%% pdf-tiedostoksi, kuvasi on oltava jpg-formaatissa tai pdf-formaatissa.
%%
%% Use this if you run pdflatex and use jpg/pdf-format pictures.
%%
\usepackage{graphicx}

%% Saat pdf-tiedoston viittaukset ja linkit kuntoon seuraavalla paketilla.
%% Paketti toimii erityisen hyvin pdflatexin kanssa. 
%%
%% Use this if you want to get links and nice output with pdflatex
\usepackage[pdfpagemode=None,colorlinks=true,urlcolor=red,%
linkcolor=blue,citecolor=black,pdfstartview=FitH]{hyperref}

%% Jos et jostain syystä tykkää käyttää
%% edellistä hyperref pakettia, voit käyttää myös seuraavaa pakettia
%% (tarvitaan lähinnä url-komennon määrittämiseen ja formatoimiseen)
%%
%% Use this if you do not like hyperref package - this
%% defines url environment and formats it correctly
\usepackage{url}

%% Matematiikan fontteja, symboleja ja muotoiluja lisää, näitä tarvitaan usein 
%%
%% Use this if you write hard core mathematics, these are usually needed
\usepackage{amsfonts,amssymb,amsbsy}  

%% Vaakasuunnan mitat, ÄLÄ KOSKE!
\setlength{\hoffset}{-1in}
\setlength{\oddsidemargin}{35mm}
\setlength{\evensidemargin}{25mm}
\setlength{\textwidth}{15cm}
%% Pystysuunnan mitat, ÄLÄ KOSKE!
\setlength{\voffset}{-1in}
\setlength{\headsep}{7mm}
\setlength{\headheight}{1em}
\setlength{\topmargin}{25mm-\headheight-\headsep}
\setlength{\textheight}{23cm}


%%\titleformat{\section}%
%%\renewcommand{\thesubsection}{\arabic{subsection}}%... from subsections


\setcounter{secnumdepth}{5}
\setcounter{tocdepth}{5}
%% Kaikki mikä paperille tulostuu, on tämän jälkeen
%%
%% Output starts here
\begin{document}

%% Korjaa vastaamaan korkeakouluasi, jos automaattisesti asetettu nimi on 
%% virheellinen 
%%
%% Change the school field to describe your school if the autimatically 
%% set name is wrong
% \university{aalto University}{aalto-Yliopisto}
% \school{School of Electrical Engineering}{SähköTekniikan korkeakoulu}

%% Vain kandityölle: Korjaa seuraavat vastaamaan koulutusohjelmaasi
%%
%% Only for B.Sc. thesis: Choose your degree programme. 
\degreeprogram{Electronics and electrical engineering}{Elektroniikka ja sähkötekniikka}
%%

%% Vain DI/M.Sc.- ja lisensiaatintyölle: valitse laitos, 
%% professuuri ja sen professuurikoodi. 
%%
%% Only for M.Sc. and Licentiate thesis: Choose your department,
%% professorship and professorship code. 
%\department{Department of Radio Science and Technology}%
%{Radiotieteen ja -tekniikan laitos}
%\professorship{Circuit theory}{Piiriteoria}
%\code{S-55}
%%

%% Valitse yksi näistä kolmesta
%%
%% Choose one of these:
\univdegree{BSc}
%\univdegree{MSc}
%\univdegree{Lic}

%% Oma nimi
%%
%% Should be self explanatory...
\author{Peter Tapio}

%% Opinnäytteen otsikko tulee vain tähän. Älä tavuta otsikkoa ja
%% vältä liian pitkää otsikkotekstiä. Jos latex ryhmittelee otsikon
%% huonosti, voit joutua pakottamaan rivinvaihdon \\ kontrollimerkillä.
%% Muista että otsikkoja ei tavuteta! 
%% Jos otsikossa on ja-sana, se ei jää rivin viimeiseksi sanaksi 
%% vaan aloittaa uuden rivin.
%% 
%% Your thesis title. If the title is very long and the latex 
%% does unsatisfactory job of breaking the lines, you will have to
%% break the lines yourself with \\ control character. 
%% Do not hyphenate titles.
\thesistitle{Selecting the right video conferencing framework for a distributed project}

\place{Espoo}
%% Kandidaatintyön päivämäärä on sen esityspäivämäärä! 
%% 
%% For B.Sc. thesis use the date when you present your thesis. 
\date{XX.05.2014}

%% Kandidaattiseminaarin vastuuopettaja tai diplomityön valvoja.
%% Huomaa tittelissä "\" -merkki pisteen jälkeen, 
%% ennen välilyöntiä ja seuraavaa merkkijonoa. 
%% Näin tehdään, koska kyseessä ei ole lauseen loppu, jonka jälkeen tulee 
%% hieman pidempi väli vaan halutaan tavallinen väli.
%%
%% B.Sc. or M.Sc. thesis supervisor 
%% Note the "\" after the comma. This forces the following space to be 
%% a normal interword space, not the space that starts a new sentence. 
\supervisor{Prof.\ Markus Turunen}{Prof.\ Markus Turunen}

%% Kandidaatintyön ohjaaja(t) tai diplomityön ohjaaja(t)
%% 
%% B.Sc. or M.Sc. thesis advisors(s). 
%%
%% Note that there has been a change in the official EN translation
%% of the Finnish title ``ohjaaja'' which in the previous version (1.5) 
%% of this document was called ``instructor''. The recommended
%% translation is now ``advisor''.  
%% However, the LaTeX internal variable remains \instructor
%% as there is little point to change the variable name. 
%%
\instructor{Official titles needed from everyone.}{Tarvitaan kaikkien tittelit vielä.}%
%\instructor{Maria Clavert}{Maria Clavert}%
%\instructor{Joona Kurikka}{Joona Kurikka}%
%\instructor{Elina Kahkonen}{Elina Kahkonen}%
%\instructor{Tuomas Paloposki}{Tuomas Paloposki}%

%% Aaltologo: syntaksi:
%% \uselogo{aaltoRed|aaltoBlue|aaltoYellow|aaltoGray|aaltoGrayScale}{?|!|''}
%% Logon kieli on sama kuin dokumentin kieli
%%
%% Aalto logo: syntax:
% \uselogo{aaltoRed|aaltoBlue|aaltoYellow|aaltoGray|aaltoGrayScale}{?|!|''}
%% Logo language is set to be the same as the document language.
\uselogo{aaltoBlue}{?}

%% Tehdään kansilehti
%%
%% Create the coverpage
\makecoverpage

%% English abstract, uncomment if you need one. 
%% 
%% Abstract keywords
\keywords{webrtc, teleconferencing,\\ product, development}
%% Abstract text
\begin{abstractpage}[english]

 This bachelors thesis is based on a multidisciplinary project based pilot course, done partially at CERN with the aim to develop new technologies for education. The course objectives and the mentoring group TALENT are presented, briefly with focus on their relevance to the work. The findings and point of view of the multidisciplinary team and the relevance of those findings to the final technical result are explained and evaluated. The final prototype, related to teleconferencing, is explained and the technical properties are analyzed.
 
\end{abstractpage}

%% Note that 
%% if you are writting your master's thesis in English place the English
%% abstract first followed by the possible Finnish abstract
%% Suomenkielinen tiivistelmä
%% 
%% Finnish abstract
%%
%% Tiivistelmän avainsanat
\keywords{webrtc, videoneuvottelu,\\ poikkitieteellinen, tuotekehitys}
%% Tiivistelmän tekstiosa
\begin{abstractpage}[finnish]
  
  Tiivistelmä käännetään kunhan se lähenee lopullista muotoaan.
  
  
\end{abstractpage}

%% Pakotetaan uusi sivu varmuuden vuoksi, jotta 
%% mahdollinen suomenkielinen ja englanninkielinen tiivistelmä
%% eivät tule vahingossakaan samalle sivulle
%%
%% Force new page so that English abstract starts from a new page
\newpage
%


%% Esipuhe 
%%
%% Preface
%\mysection{Esipuhe}
\mysection{Preface}
I'd like to thank...

%Haluan kiittää Professori Pirjo 
%Professoria ja ohjaajaani Olli Ohjaajaa hyvästä ja 
%huonosta ohjauksesta.\\

\vspace{5cm}
Otaniemi, 9.3.2014

\vspace{5mm}
{\hfill Peter W.\ Tapio \hspace{1cm}}

%% Pakotetaan varmuuden vuoksi esipuheen jälkeinen osa
%% alkamaan uudelta sivulta
%%
%% Force new page after preface
\newpage


%% Sisällysluettelo
%% addcontentsline tekee pdf-tiedostoon viitteen sisällysluetteloa varten
%% 
%% Table of contents. 
%\addcontentsline{toc}{section}{Sisällysluettelo}
\addcontentsline{toc}{section}{Contents}
%% Tehdään sisällysluettelo
%%
%% Create it. 
\tableofcontents


%% Symbolit ja lyhenteet
%%
%% Symbols and abbreviations
%\mysection{Symbolit ja lyhenteet}
\mysection{Abbreviations}

\begin{tabular}{ll}
CERN        & European Organization for Nuclear Research \\
CBI         & Challenge Based Innovaition course at CERN,\\
            & in collaboration with various universities \\
TCP         & Transfer Control Protocol \\ %%
UDP         & User Datagram Protocol \\
HTML5       & Hyper Text Markup Language version 5 \\
WebRTC      & Web Real Time Communication \\
IETF        & The Internet Engi- neering Task Force \\
W3C         & World Wide Web Consortium \\
API         & Application Programming Interface \\
RTP         & Real-Time Transport Protocol \\
IT          & Infromation Technology \\
NLP         & Natural Language Processing \\
IBL         & ATLAS Insertable B-Layer detector 
\end{tabular}


%% Sivulaskurin viilausta opinnäytteen vaatimusten mukaan:
%% Aloitetaan sivunumerointi arabialaisilla numeroilla (ja jätetään
%% leipätekstin ensimmäinen sivu tyhjäksi, 
%% ks. alla \thispagestyle{empty}).
%% Pakotetaan lisäksi ensimmäinen varsinainen tekstisivu alkamaan 
%% uudelta sivulta clearpage-komennolla. 
%% clearpage on melkein samanlainen kuin newpage, mutta 
%% flushaa myös LaTeX:n floatit 
%% 
%% Corrects the page numbering, there is no need to change these
\cleardoublepage
\storeinipagenumber
\pagenumbering{arabic}
\setcounter{page}{1}


%% Leipäteksti alkaa
%%
%% Text body begins. Note that since the text body
%% is mostly in Finnish the majority of comments are
%% also in Finnish after this point. There is no point in explaining
%% Finnish-language specific thesis conventions in English.
%\section{Johdanto}
\section{Introduction}

%% Ensimmäinen sivu tyhjäksi
%% 
%% Leave first page empty
\thispagestyle{empty}

This work is based on a project done at the European Organization for Nuclear Research (CERN) for the TALENT project, that is a part of the Marie Curie Initial Training Network. The purpose of this thesis is to elaborate the background of the prototype produced for the TALENT project, to explain how the needs of the user affected the selection of the technological solution and elaborate on the input given by the multidisciplinary team members assigned to the project.
The technologies and standards relevant to the project will be explored and evaluated against the findings of the student project as well as scientific sources. The technological reasons for selecting the prototype platform are explained. A brief account of other limitations to the project is also given. The strengths and weaknesses of the selected technology will be listed and other possible solutions will be presented for comparison.


%% Opinnäytteessä jokainen osa alkaa uudelta sivulta, joten \clearpage
%%
%% In a thesis, every section starts a new page, hence \clearpage
\clearpage

%\section{Aikaisempi tutkimus}
\section{The CBI project for TALENT at CERN}

This chapter briefly explains the nature of the Challenge Based Innovation -course and its relation to CERN and the TALENT project. The connection to research and development activities at CERN is an important influence on the decisions made during the project and provides an important point of view also in this work. Some detail is given about the nature of the course and the methods used. The partners involved in the course and the project are introduced. 


\subsection{The Challenge Based Innovation course}

The Challenge Based Innovation (CBI) course is developed and organized in collaboration with CERN and Aalto University in Finland, University of Modena and Reggio Emilia (UNIMORE) in Italy and National Technical University of Athens (NTU) in Greece. CBI is a project based course in product development, that utilizes the Design Thinking process to guide multidisciplinary teams in prototyping. CBI is based in CERN, more specifically under a platform called IDEA\textsuperscript{S}. IDEA\textsuperscript{S} is an experimental platform that aims to support developing emerging technologies from CERN. The course started end of October at CERN with a two week intensive training period and team forming. During the course altogether four intensive periods were held, with options to travel to CERN and increased communication within the teams. The last two week visit to CERN ended in the CBI presentations on 7.3. and mostly consisted of building the final prototypes.

\subsection{The project team}

The project team was selected from the applicants and formed by the course staff. For the course to work as intended and to have the right kind of team dynamics, the teams were mixed so that each of the two teams on the course had some students with skills in engineering, industrial design and business. The talent team had the following members from various universities and countries.

\begin{itemize}
\item[--] Maria Stathi, Architecture, National Technical University of Athens (NTU), Greece
\item[--] Eleonora Forghieri, Management Engineering, University of Modena and Reggio Emilia (UNIMORE), Italy
\item[--] Valeria Fortunato, University of Modena and Reggio Emilia (UNIMORE), Italy
\item[--] Luca Morini, Mechatronics, University of Modena and Reggio Emilia (UNIMORE), Italy
\item[--] Jani Nurmi, Finance, Aalto University, Finland
\item[--] Aleksanteri Heliövaara, Industrial Design, Aalto University, Finland
\item[--] Hannes Kallioinen, Mechanical Engineering, Aalto University, Finland
\item[--] Peter Tapio, Electronics, Aalto University, Finland
\end{itemize}

\subsection{The TALENT project}

The TALENT project at CERN, has young researchers in the field of instrumentation for radiation detection, developing new state--of--the--art technologies for the ATLAS detector. The project involves people from various countries and universities, working with multidisciplinary industry in the various fields of advanced radiation sensors, fast and efficient data acquisition electronics, new cooling technologies and ultra light support structures.



\subsection{The project brief for the CBI team}

A project brief was written before the beginning of the course, in order to set a direction for the student project and help the involved parties agree and understand what will be done during the course. This is a well established method used when commissioning a project. The brief defines the scope of the project and it can be made narrow or open, in this case the initial brief was left very open.

The two main requirements for the project were a connection to TALENT and a relation to education. The brief was left fairly open, so that the multidisciplinary team could customize it to suit their specific set of skills. On a more detailed level, three different tracks were presented for the team to choose.

\begin{itemize}
\item[--]Using TALENT technologies for education.
\item[--]Educating a focus group about the TALENT technologies and the science behind the various technologies.
\item[--]To develop a technology that supports the research activities of TALENT.
\end{itemize}

The final problem definition and type of solution was left for the student team to decide and discuss with members of the TALENT project. A design thinking approach would then be used to further determine the final nature of the project.

\clearpage

\section{Scope of the project and reasoning for the final concept}

Of the available options for directions that the project could take, all were investigated and some prototypes were produced for each scenario. Using TALENT technology for the development of education was the most technical in nature, because of the advanced nature of the technologies, but this also limited the scope of the options. The highly specialized nature of the technology for radiation detection, is best suited for applications closely involved with radiation. Outside of the obvious topics of medical and nuclear education, some sensor and safety related technologies were considered, but the scope was found to be very limited in each scenario.

Educating people about TALENT technology and its use in detecting radiation or demonstrating the nature of radiation, was a very popular subject within the team and received positive feedback from TALENT researchers as well. Ideas for this track had a very even balance of industrial design, business sense and technical challenges. Some concepts for this track still remain good choices for future development, but they were discarded by the team for lack of originality and the kind of massive impact the team was looking to find.

After the first major divergent phase of the project, a track was chosen for the project, TALENT would be the client and the focus would be on remote participation and information management. Developing for TALENT gave the team the most freedom to develop a comprehensive system that could also be adapted to a more traditional classroom setting. The team wanted to produce a concept that could have a wide impact on the world of education and the way people learn in the future. 


\subsection{Needfinding}

Several experts and teachers were interviewed in Finland, Italy and Greece, to gain an understanding into technology used in the classrooms of today. The interviews were planned, coordinated and qualitative in nature, but the main emphasis was in information gathering to support the teams ideation, instead of actual scientific research. Members in each country, mapped their resources and the availability of subjects for the interviews. Some general surveys were also done online, distributed to students and local contacts through online services. We also gathered interesting articles about the latest developments in education and the progressive efforts done around the world by professors and professionals in the field. Interesting stories were followed to their source and further deconstructed in team meetings. Due to the relatively short time we had for the research, not all of the contacted people had a chance to reply, but the ones that did were very enthusiastic about our approach.

Activities specifically related to TALENT were performed after the decision to use TALENT as the target group was solidified. Members of the project team observed TALENT coordination meetings and engaged in creative methods with members of TALENT. Regular meetings were also arranged when team members were present at CERN. Starting from the general information accumulated during the initial research the idea for the final concept began to form.

Shadowing TALENT members and the discussions with the researchers helped the team find needs that TALENT has. As TALENT researchers develop technologies and components for the Insertable B--Layer for the ATLAS detector, they have to manage a massive amount of information and can afford virtually no mistakes in communication, as all the components must fit together precisely. The large elements are designed by globally distributed teams, each working on different parts of what is essentially a single device. Since the detector layer is built in parts but is practically not a modular device, the requirements for each component in the system are very rigid. Great care must be taken, not only in the design of the components, but in the assembly of the final product as well. Errors in the assembly sequence can lead to delays of several weeks or months, as parts may have to be redone. The distributed nature of the TALENT project also means that the IT infrastructure is fragmented into what ever each university, company or organization uses. Due to the difference in IT infrastructures, special care must be put into making the prototype compatible with the various platforms. 




\subsection{Final concept}

The final concept prof the CBI project inherited many qualities from the intermediate prototypes that we had made during the explorative phase of the project. From concepts of holographic displays and descriptive visualizations, we took the elements of futuristic visualizations. From the Greek concept of the learning fountain, which was a fountain like element designed to enable people to consume information, we took some industrial design aspects and the idea of having a central hub. The concept contains some physical elements that are designed to naturally convey information and help the users on remote interactions beyond audio and video. A large touchscreen table could be used as a hub element in the concept to aid in manipulating structured data and help the users share conceptual creations through drawing on the surface or scanning in their meeting notes. These features were selected and tailored to answer the discovered needs of the users.

The final concept is a system to aid in project management and information handling. The system would aggregate project related meta level information, files and produced data such as audio recordings, and provide an interface to intuitively browse and search the accumulated data. By providing a system for presentations, teleconferencing and remote communication, the system can gather meta data and create a relational graph by analyzing the data. Keywords and topics can be extracted from presentation files and mined from material created by speech--to--text algorithms. User accounts carry information of organizations, projects and professional skills related to the user, which can then be associated to their actions and to meetings they participate attend.

The information can then be displayed easily on a timeline, which is the default view for the system. Displaying the information as a connected graph instead of a timeline, allows the user filter the information according to the type of connections, thus allowing the user to find information by its qualities, instead of literal text content. The semantic search can be implemented by using Natural Language Processing (NLP) algorithms to extract named entities, such as dates, places, names or locations.

\clearpage

\section{Final prototype for the concept}

As a final prototype that demonstrates core functions of the overall concept, the team decided to implement a teleconferencing system, that can implement speech--to--text conversion on the processed audio and is compatible with analytics to extract useful information. It would need to be possible to implement all the required functionality in the few weeks of time that were left when the concept was finally locked. Even with the short time span, the team wanted to make the system in a way that it might be possible to test it more extensively on various platforms. 

\subsection{Requirements}
- Practical matters
- User centered design

In this case the system needed to demonstrate that multiple users can communicate remotely, as they would with any other system, but so that we can add an unobtrusive layer of analytics on top of the teleconferencing platform. If the users would need to take extra action to enable the analytics or participate in the administration of the system, it would raise the barrier of using the system and have a negative impact on the actual communication that needs to happen. 


\subsubsection{Technical limitations}

- Network limitations
- Bandwidth
- Latency
- Platform compatibility

The system also needs to be usable by anyone participating in the project, without the need to learn new comlicated programs or a need to install proprietary software. Support for Linux, OS X and Windows should be implemented, so that the system would work regardless of the platform that the participating organization is using. The system should not suffer from losing connection to the Internet, meaning that local functions should stay available even in the event that the outside network is not reachable. The local services still provide recording and storage capabilities that can be synchronized with outside systems once connectivity has been returned.


\subsubsection{Other limitations}

- Time to prototype
- Resources
- Programmers

Other practical limitations to implementing the prototype, also had a significant impact on the features of the final prototype. When the scope and features of the prototype were decided, there was about one month left to build the server and program the needed components. This meant that the solution needed to be relatively easy to implement and that it would make sense to build it gradually, building features on top of each other. Other material also needed to be produced at the same time to be able to present everything in the final event. Since there was only one member in the team with experience and the required knowledge in programming, it limited the available options for task division. 

\subsubsection{Selected client--side software}
(Chrome, HTML5, javascript)
- The team selected to use an experimental open API in a popular web browser, as the basis for the teleconferencing application. 
- HTML5 offers advanced media capabilities, natively supporting audio and video tags, that make it possible to easily embed media content onto web pages.
- Javascript is a prototype based scripting language, that is easy to understand and more likely usable by the teams limited amount of programmers because of its use as the client--side scripting language on the world wide web.


\subsubsection{Selected server--side software}

As most of the main features were implemented in the client--side, the server--side software just needed to be simple to operate and powerful enough to serve all the content required by the system. An HTTP server is required to provide static content to the user and act as the application provider. The application provider in this context is the server that hosts the client application that is server to the users browser. The server is the initial point of contact between the peers that are connecting to the system. Nginx was selected as the HTTP server software for this projects purposes. Nginx is an HTTP server and forwarding proxy programmed by Igor Sysoev (\url{http://nginx.org}).

In order for the client applications to communicate with each other and update the dynamically added content to the users screen, a server--side scripting platform was needed to handle the messages and provide the necessary sockets for communicating. NodeJS (\url{http://nodejs.org}) is a Javascript based server--side platform that enables the building of highly scalable and distributed applications. NodeJS is memory efficient under high load and runs on a very small footprint, which in this case was important because the prototype would be run locally on cheap hardware. The platform is also built in a way that very few function calls block, so that most systems will respond fast, which reduces the latency in a messaging system.



\subsection{WebRTC}

WebRTC stands for Web Real Time Communication and it is a set of open standards being defined by the Internet Engineering Task Force (IETF) and the World Wide Web consortium (W3C). It is being developed to provide a set of standard application programming interfaces (API) to replace the currently used proprietary protocols and plugins. The core components of WebRTC are peer--to--peer audio and video streaming and a multiplexing peer--to--peer data transfer channel. The web server hosting the application, will relay the communication between the peers until they have established a direct channel using the RTCPeerConnection API call. The RTCPeerConnection \cite{Jennings}

The standard will define mandatory codecs to guarantee compatibility for audio and video compression, but developers are free to add support for more codecs. At the time of writing this thesis, the supported codecs were OPUS and G.711 for audio, and V8 and H.264 for video. It is 
\cite{Jennings}

\subsubsection{WebRTC Architecture}

(insert graph of WebRTC architecture, illustrating the communication between the server and the peers, before and after communication is established between peers)

The above graph illustrates the architecture of a WebRTC based system, where an application provider serves a Javascript application to each peer. The application uses the common WebRTC API that is built into the client (e.g. browser). The application is then able to request access to the local microphone or video feed, using the GetUserMedia API call, which produces one or more sychronized streams, according to what is requested. The peers are then able to call for an RTCPeerConnection and add their local media stream to it. Each call to RTCPeerConnection triggers a negotiation between the peers to agree on bandwidth limitations, format and resolution. \cite{Jennings} 

The peer-to-peer nature of WebRTC means that a separate connection is created for each peer using the same application, this limits the maximum amount of peers to what a client connection can support. Conferences or media streaming in a one to many --topology can be achieved by using a media processing node that handle the mixing of media between RTCPeerConnections. \cite{Jennings} In this prototype setting the media mixing server was not required for the proof of concept phase, but remains available for implementation if the systems is ever developed further.

\clearpage


\section{Conclusion}

The conclusion will heavily depend on the yet to be read scientific publications.

The development of advanced media capabilities into web browsers has made it possible for developers to rapidly develop interactive real--time applications. In the related CBI project, the WebRTC platform was chosen because of its versatility, easy to implement API and compatiblility with multiple platforms, including browsers for mobile devices. It enabled the prototype to be tested in many environments and eventually would help in the adoption of the final system, because the system does not require the installation of any additional software by the user.

The system includes bandwidth management and supports common codecs. 

\clearpage

%% Lähdeluettelo
%\addcontentsline{toc}{section}{Viitteet}
\addcontentsline{toc}{section}{References}

\begin{thebibliography}{99}

\bibitem{Adeyeye} Adeyeye,\ M., Makitla,\ I., Fogwill,\ T. (2013) \textit{Determining the signalling overhead of two common WebRTC methods: JSON via XMLHttpRequest and SIP over WebSocket} IEEE 2013

\bibitem{Jennings} Jennings,\ C., Hardie,\ T., Westerlund, M. (2013) \textit{Real-time communications for the web} Communications Magazine, IEEE Vol. 51, Iss. 4, pp. 20-26 

\bibitem{Garaizar} Garaizar,\ P., Vadillo,\ M.\ A., Lopez-de-Ipina,\ D. (2012) \textit{Benefits and pitfalls of using HTML5 APIs for online experiments and simulations} Remote Engineering and Virtual Instrumentation (REV), 2012 9th International Conference on pp. 1-7. ISBN: 978-1-4673-2540-0

\bibitem{Hunger} Hunger,\ A., Hirlehei,\ A. (2013) \textit{Supporting globally distributed work - Cultural adaptivity meets groupware tailorability} Lecture Notes in Computer Science (including subseries Lecture Notes in Artificial Intelligence and Lecture Notes in Bioinformatics) 8024 LNCS: (PART 2) p. 219-227 2013. ISBN: 9783642391361

\bibitem{Mootee} Mootee,\ I. (2013) \textit{Design thinking for strategic innovation : what they can't teach you at business or design school} Hoboken: Wiley , 2013. (e-book) ISBN: 9781118748855

\bibitem{} \textit{}


\end{thebibliography}


\end{document}


